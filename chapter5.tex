\chapter{超对称}
\section{标准模型}
这本质上是唯象学的内容,但是超对称的提出很大程度上就是为了寻找超出标准模型的物理。标准模型从群论上看由三个规范群描述:
\begin{equation}
	\mathrm{SU}(3)_\mathrm{C}\times\mathrm{SU}(2)_\mathrm{L}\times\mathrm{U}(1)_\mathrm{Y}
\end{equation}
传递相互作用的规范玻色子在其自伴表示下,而参与构成物质的费米子处于其基本表示下,其实这一点就很不自然,我们是从为了解释实验数据而要求费米子处于基本表示,原则上来说理论允许费米子处于任意其他表示之中。

规范玻色子本身是没有质量的,但是$W^\pm/Z^0$玻色子有质量,这可以通过引入一个复标量场,Higgs,通过$\mathrm{SU}(2)_\mathrm{L}\times\mathrm{U}(1)_\mathrm{Y}\to \mathrm{U}(1)_{\mathrm{EM}}$的对称性自发破缺带来质量项,还剩下一个实标量场自由度和矢量场自由度没有破缺,它们构成Higgs粒子和光子。其它费米子质量的起源也可以通过和Higgs引入Yukawa相互作用项耦合,自发破缺后带来质量项。注意QCD中费米子quark的质量项是可以直接通过在$\mathcal{L}_{\mathrm{QCD}}$中添加正定的质量项得到,但是由于电弱规范理论是一个手征理论(注意$\mathrm{SU}(2)_\mathrm{L}$下标$L$),所以直接添加质量项会破缺手征性,质量项只能通过Higgs机制得到,详细的推导可见\href{https://yzhxxzxy.github.io/teaching/2209_SM_FR.pdf}{余钊焕老师的讲义}。也正是因为有这么个更大的群到子群的对称性自发破缺,所以电弱理论已经统一,但是QCD还单独落在外面,大统一理论(GUT)的目标就是找到更大的群如$SU(5)$,让它自发破缺到$\mathrm{SU}(2)_\mathrm{L}\times\mathrm{U}(1)_\mathrm{Y}\to \mathrm{U}(1)_{\mathrm{EM}}$,自然得到三种基本相互作用。
\subsection{标准模型拉氏量}
我们先给出一份网上流传甚广的“物理学最复杂公式”,也就是标准模型的Lagrangian,首先是QCD部分,动力学为:\footnote{选取规定\[D_\mu=\partial_\mu-\mathrm{i}g_\mathrm{s}G_\mu^at^a,\quad G^{a\mu\nu}\equiv\partial^\mu G^{a\nu}-\partial^\nu G^{a\mu}+g_\mathrm{s}f^{abc}G^{b\mu}G^{c\nu}\]}
\begin{equation}
	\begin{aligned}
		\mathcal{L}_\mathrm{QCD}=&\sum_q\bar{q}(\mathrm{i}\gamma^\mu D_\mu-m_q)q-\frac14G_{\mu\nu}^aG^{a\mu\nu},\quad q=u,d,s,c,b,t,\quad a=1,\cdots,8\\
		=&\sum_q[\bar{q}(\mathrm{i}\gamma^{\mu}\partial_{\mu}-m_{q})q+g_{\mathrm{s}}G_{\mu}^{a}\bar{q}\gamma^{\mu}t^{a}q]+\frac12[(\partial_{\mu}G_{\nu}^{a})(\partial^{\nu}G^{a\mu})-(\partial_{\mu}G_{\nu}^{a})(\partial^{\mu}G^{a\nu})]\\
		&-g_{\mathrm{s}}f^{abc}(\partial_{\mu}G_{\nu}^{a})G^{b\mu}G^{c\nu}-\frac{1}{4}g_{\mathrm{s}}^{2}f^{abc}f^{ade}G_{\mu}^{b}G_{\nu}^{c}G^{d\mu}G^{e\nu}.
	\end{aligned}
\end{equation}
然后需要引入FP鬼场量子化方法固定规范,还要加入两项:
\begin{equation}
	\mathcal{L}_{\mathrm{QCD,GF}}=-\frac1{2\xi}(\partial^{\mu}G_{\mu}^{a})^{2}
\end{equation}
和鬼场:
\begin{equation}
	\mathcal{L}_{\mathrm{QCD,FP}}=-\bar{\eta}_{g}^{a}\left(g_{\mathrm{s}}\frac{\delta G^{a}}{\delta\alpha^{c}}\right)\eta_{g}^{c}=-\bar{\eta}_{g}^{a}(\delta^{ac}\partial^{2}+g_{\mathrm{s}}f^{abc}\partial^{\mu}G_{\mu}^{b})\eta_{g}^{c}\rightarrow-\bar{\eta}_{g}^{a}\delta^{ab}\partial^{2}\eta_{g}^{a}+g_{\mathrm{s}}f^{abc}(\partial^{\mu}\eta_{g}^{a})G_{\mu}^{b}\eta_{g}^{c}.
\end{equation}
这里鬼场是Grassmannian。电弱规范理论就麻烦很多,自发破缺后得到真正和粒子对应的场要由自发破缺前的场通过适当的线性组合得到,这里只给最终结论。首先是希格斯机制带来的规范玻色子质量项
\begin{equation}
	\mathcal{L}_\mathrm{GBM}=m_W^2W^{+\mu}W_\mu^-+\frac12m_Z^2Z^\mu Z_\mu 
\end{equation}
然后是Yukawa耦合带来的Higgs粒子和费米子质量项:
\begin{equation}\label{eq:5.6}
	\mathcal{L}_{\mathrm{Y}}=-m_{d_i}\bar{d}_id_i-m_{u_i}\bar{u}_iu_i-m_{\ell_i}\bar{\ell}_i\ell_i-\frac{m_{d_i}}vH\bar{d}_id_i-\frac{m_{u_i}}vH\bar{u}_iu_i-\frac{m_{\ell_i}}vH\bar{\ell}_i\ell_i
\end{equation}
这里$i$隐含对三代quark求和。费米子和规范场之间通过协变导数项引入相互作用,或者说等价于和一堆$U(1)$的流耦合:
\begin{equation}
	\mathcal{L}_{\mathrm{EWF}}\supset A_{\mu}J_{\mathrm{EM}}^{\mu}+Z_{\mu}J_{Z}^{\mu}+W_{\mu}^{+}J_{W}^{+,\mu}+W_{\mu}^{-}J_{W}^{-,\mu}
\end{equation}
其中:
\begin{equation}
	\begin{gathered}
		J_\mathrm{EM}^\mu\equiv\sum_fQ_fe\bar{f}\gamma^\mu f\\
		\begin{aligned}J_Z^\mu&\equiv\frac g{2c_\mathrm{W}}\sum_f\bar{f}\gamma^\mu(g_\mathrm{V}^f-g_\mathrm{A}^f\gamma^5)f=\frac g{c_\mathrm{W}}\sum_f(g_\mathrm{L}^f\bar{f}_\mathrm{L}\gamma^\mu f_\mathrm{L}+g_\mathrm{R}^f\bar{f}_\mathrm{R}\gamma^\mu f_\mathrm{R})\end{aligned}\\
		J_{W}^{+,\mu}\equiv\frac g{\sqrt{2}}(\bar{u}_{i\mathrm{L}}\gamma^{\mu}V_{ij}d_{j\mathrm{L}}+\bar{\nu}_{i\mathrm{L}}\gamma^{\mu}\ell_{i\mathrm{L}}),\quad J_{W}^{-\mu}\equiv(J_{W}^{+\mu})^{\dagger}=\frac g{\sqrt{2}}(\bar{d}_{j\mathrm{L}}V_{ji}^{\dagger}\gamma^{\mu}u_{i\mathrm{L}}+\bar{\ell}_{i\mathrm{L}}\gamma^{\mu}\nu_{i\mathrm{L}})
	\end{gathered}
\end{equation}
这里$f$表示任意的费米子,$\ell$表示轻子,$u,d$表示夸克,$\nu$是中微子。$SU(2)\times U(1)$的规范场自相互作用可以由动能项:
\begin{equation}
	\mathcal{L}_\mathrm{EWG}=-\frac14W_{\mu\nu}^aW^{a\mu\nu}-\frac14B_{\mu\nu}B^{\mu\nu},
\end{equation}
来导出,注意这里的$W^a,B$都是未破缺之前的场,破缺后的场需要线性组合得到$W^{\pm}/Z^0,A^\mu$,经过贼复杂的计算后得到:
\begin{equation}
	\begin{aligned}
		\mathcal{L}_{\mathrm{EWG}}= & \frac{1}{2}\left[\left(\partial_{\mu} A_{\nu}\right)\left(\partial^{\nu} A^{\mu}\right)-\left(\partial_{\mu} A_{\nu}\right)\left(\partial^{\mu} A^{\nu}\right)\right]+\frac{1}{2}\left[\left(\partial_{\mu} Z_{\nu}\right)\left(\partial^{\nu} Z^{\mu}\right)-\left(\partial_{\mu} Z_{\nu}\right)\left(\partial^{\mu} Z^{\nu}\right)\right] \\
		& +\left(\partial_{\mu} W_{\nu}^{+}\right)\left(\partial^{\nu} W^{-\mu}\right)-\left(\partial_{\mu} W_{\nu}^{+}\right)\left(\partial^{\mu} W^{-\nu}\right)+\frac{g^{2}}{2}\left(W_{\mu}^{+} W^{+\mu} W_{\nu}^{-} W^{-\nu}-W_{\mu}^{+} W^{+\nu} W_{\nu}^{-} W^{-\mu}\right) \\
		& +\mathrm{i} e\left[\left(\partial_{\mu} W_{\nu}^{+}\right) W^{-\mu} A^{\nu}-\left(\partial_{\mu} W_{\nu}^{+}\right) W^{-\nu} A^{\mu}-W^{+\mu}\left(\partial_{\mu} W_{\nu}^{-}\right) A^{\nu}+W^{+\nu}\left(\partial_{\mu} W_{\nu}^{-}\right) A^{\mu}\right. \\
		& \left.\quad+W^{+\mu} W^{-\nu}\left(\partial_{\mu} A_{\nu}\right)-W^{+\nu} W^{-\mu}\left(\partial_{\mu} A_{\nu}\right)\right] \\
		& +\mathrm{i} g c_{\mathrm{W}}\left[\left(\partial_{\mu} W_{\nu}^{+}\right) W^{-\mu} Z^{\nu}-\left(\partial_{\mu} W_{\nu}^{+}\right) W^{-\nu} Z^{\mu}-W^{+\mu}\left(\partial_{\mu} W_{\nu}^{-}\right) Z^{\nu}+W^{+\nu}\left(\partial_{\mu} W_{\nu}^{-}\right) Z^{\mu}\right. \\
		& \left.\quad+W^{+\mu} W^{-\nu}\left(\partial_{\mu} Z_{\nu}\right)-W^{+\nu} W^{-\mu}\left(\partial_{\mu} Z_{\nu}\right)\right] \\
		& +e^{2}\left(W_{\mu}^{+} W^{-\nu} A_{\nu} A^{\mu}-W_{\mu}^{+} W^{-\mu} A_{\nu} A^{\nu}\right)+g^{2} c_{\mathrm{W}}^{2}\left(W_{\mu}^{+} W^{-\nu} Z_{\nu} Z^{\mu}-W_{\mu}^{+} W^{-\mu} Z_{\nu} Z^{\nu}\right)\\
		&+egc_{\mathrm{W}}(W_{\mu}^{+}W^{-\nu}A_{\nu}Z^{\mu}+W_{\mu}^{+}W^{-\nu}A^{\mu}Z_{\nu}-2W_{\mu}^{+}W^{-\mu}A_{\nu}Z^{\nu})
	\end{aligned}
\end{equation}

把前面讨论的这一堆拉氏量全部凑在一堆就是标准模型了!但实际计算上我们不会真的取考虑全部的标准模型,往往是对一部分模型积掉实验能标上更高能的自由度得到一个有效理论,比如$\pi$介子理论,但是也够难算的了。
\subsection{标准模型的局限}

虽然标准模型在解释实验现象上取得了巨大的成功,但是仍有非常多的问题亟待解决,关于唯象上新物理的寻找的更多内容可见书籍\cite{baer_weak_2006}的前两章。
\begin{description}
	\item[参数过多] 即使是标准模型本身,就需要用19个实验参数去描述,而实验参数越多意味着这个理论本身越想是一个没搞清楚内部结构的黑箱。况且冯诺依曼也说过:“四个参数画大象,五个参数鼻子动”\cite{10.1119/1.3254017}
	\item[没有引力]引力可以量子化,至少在平直时空背景下我们可以微扰地去做量子化。对于任何一个量子场论,都可以在有效场论的框架下写成$\mathcal{L}=\sum_ic_i\mathcal{O}_i$的形式,这些$c_i$是需要计算散射振幅后通过实验拟合得到的,或者说理论的自由度。在低能标下绝大部分的自由度都是看不到的,而实验能标越高,我们也愈发要往里面加新的参数,做新的实验,对于QED,QCD,在$\Lambda_{\mathrm{QCD}}$能标以下做实验确定的参数可以通过重整化群流方法得到高能标下的其它自由度,一直推到紫外也可以,意味着我们不用每个能标都做一次实验去确定参数,只用在特定能标以下确定有限多个参数就行了,这样的理论是紫外完备的,也就是可重整的。但是引力理论并不是可重整的,也就是说每当升高能标,理论中就会不断出现新的自由度,而且还不能用前面的实验来确定,只能做新的实验,这样每个能标我们都要确定一次参数来确定这个能标下的量子引力理论。而且引力量子化之后的引力子目前实验上也没有探测到。当然,目前来说在特定能标下我们还是能半经典地去量子化引力,并且去计算引力子的散射,注意,不可重整化不是意味着圈图的计算我们无法处理无穷大,我们仍旧可以用正规化重整化的套路去掩盖无穷大并且与实验观测拟合,只是在Wilson有效场论的框架下引力的量子理论目前不完备,必须不断地去做实验,每个能标有一套自己的量子引力理论,不能用重整化群流联系。
	\item[没有暗物质] 即使你去考虑Einstein引力本身,你也缺了点东西,天文观测有充足的表明暗物质存在,特别是冷暗物质。但标准模型中没有任何一个粒子对应暗物质。目前暗物质从粒子物理角度的解释就是去造各种新奇的粒子。见综述\cite{frieman_lectures_2008,Young:2016ala,bauer_yet_2019}
	\item[中微子震荡]从前面的\ref{eq:5.6}可以看到即便是引入对称性破缺,中微子仍然是没有质量的。这在很长的一段时间内被认为是完全正确的,知道中微子震荡的发现,暗示着中微子有质量,而且三代中微子质量顺序以及中微子是Majorana还是Dirac质量项目前都有待研究,而且直接往标准模型里面添加中微子质量项又是非常不自然的。有关更多粒子物理历史方面的论述可见{\itshape Griffiths}的书籍\cite{doi:https://doi.org/10.1002/9783527618460.ch1}
	\item [$\ldots$]$\ldots$
\end{description}

\PRLsep

目前的实验表明$\mathrm{TeV}$能标上必然有新物理,但是我们完全不清楚新物理以怎样的方式出现,超对称就在这样一个历史背景下产生了。不过后面会看到,把超对称加入标准模型后问题会变得更糟。

\section{Wess-Zumino模型}
这只是一个玩具模型,可以认为是最简单的超对称实现,但是用玩具模型可以讲清楚很多物理,比如QFT里面一般上来就会讲$\varphi^4$理论。