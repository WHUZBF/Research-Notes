%!TEX TS-program = xelatex
%!TEX encoding = UTF-8

% LaTeX source for book ``代数学方法'' in Chinese
% Copyright 2018  李文威 (Wen-Wei Li).
% Permission is granted to copy, distribute and/or modify this
% document under the terms of the Creative Commons
% Attribution 4.0 International (CC BY 4.0)
% http://creativecommons.org/licenses/by/4.0/

% 《代数学方法》卷一: 基础结构 / 李文威
% 使用自定义的文档类 AJbook.cls. 自动载入 xeCJK. 需要额外档案如下:
% font-setup-HEP.tex/font-setup-open.tex, coverpage.tex, titles-setup.tex, mycommand.sty, myarrows.sty
% 文档类选项 (key/val 格式):
% mydraft = true (未定稿, 底部显示日期) 或 false (成品), 默认 false,
% mycolors = true (链接带颜色无框) 或 false (黑色无框), 默认 true,
% coverpage = 封面档档名, 默认为空 (此时不制作封面), 若为 *.pdf 的形式则直接引入 PDF 页面.
% fontsetup = 字体设置档档名,
% titlesetup = 章节格式设置档名.

% 需动用 imakeidx + xindy (两份索引), biblatex + biber (书目). 
% 索引用土法进行中文排序: 如 \index{zhongwen@中文} 等.
\documentclass[
%	draftmark = true,
	colors = true,
%	colors = false,
	coverpage = coverpage.tex,
%	coverpage = coverpage.pdf,
	fontsetup = font-setup-open.tex,
%	fontsetup = font-setup-HEP.tex,
	titlesetup = titles-setup.tex
]{AJbook}


\usepackage[backend=bibtex, hyperref=auto, backref=true, backrefstyle=three]{biblatex}
\usepackage{mathrsfs}
\usepackage{bbm}
\usepackage{colortbl}  %彩色表格需要加载的宏包
\usepackage{xcolor}
\usepackage{simpler-wick}
\usepackage{stmaryrd} \SetSymbolFont{stmry}{bold}{U}{stmry}{m}{n}	% 避免警告 (stmryd 不含粗体故)
\usepackage{array}
\usepackage{makecell}	% 便于制表
\usepackage{tikz-cd}  % 使用 TikZ 绘图
\usetikzlibrary{positioning, patterns, calc, matrix, shapes.arrows, shapes.symbols}
\usepackage{braids}
\usepackage{tqft}
\usepackage{ytableau}
\usepackage{physics}
\usepackage{slashed}
\usepackage{float}
\usepackage{tikz}
\usepackage{tikz-feynman}
\usepackage{booktabs}
\definecolor{tyellow}{HTML}{FFD05B}
\definecolor{tyellowlight}{HTML}{FFE39D}
\definecolor{tyellowlighter}{HTML}{FFFBF0}
\definecolor{tyellowdark}{HTML}{D09B18}
\definecolor{tyellowdarker}{HTML}{715000}

\definecolor{tturq}{HTML}{3EAF7F}
\definecolor{tturqlight}{HTML}{86DAB6}
\definecolor{tturqlighter}{HTML}{EEFCF6}
\definecolor{tturqdark}{HTML}{118F59}
\definecolor{tturqdarker}{HTML}{004D2C}

\definecolor{tblue}{HTML}{3E83A1}
\definecolor{tbluelight}{HTML}{86BCD4}
\definecolor{tbluelighter}{HTML}{EEF8FC}
\definecolor{tbluedark}{HTML}{156283}
\definecolor{tbluedarker}{HTML}{033247}%定义颜色,没有管那些颜色用不到了
\usetikzlibrary{hobby, calc, intersections, decorations.markings, decorations.pathreplacing} %libraries
\tikzset{>=latex} %sets -> to render as -latex arrow tip
%https://tex.stackexchange.com/questions/54796/how-to-set-default-style-for-arrow-tips-in-tikz
% PGF plots 用于封面绘制
\usepackage{pgfplots}
\pgfplotsset{compat=newest}
\newcommand{\PRLsep}{\noindent\makebox[\linewidth]{\resizebox{0.3333\linewidth}{1pt}{$\bullet$}}\bigskip}% 生成PRL样式的分割线

% 设置章节深度
%\setcounter{secnumdepth}{1}

% 必要时仅引入部分档案
% \includeonly{}

%\usepackage{myarrows}				% 使用自定义的可伸缩箭头
%\usepackage{mycommand}				% 引入自定义的惯用的命令

% 生成索引: 选用 xindy + imakeidx
\usepackage[xindy, splitindex]{imakeidx}
\makeindex[
	columns=2,
	program=truexindy,
	intoc=true,
	options=-M texindy -I xelatex -C utf8,
	title={名词索引暨英译}]	% 名词索引
\makeindex[
	columns=3,
	program=truexindy,
	intoc=true,
	options=-M numeric-sort -M latex -M latex-loc-fmts -M makeindex -I xelatex -C utf8,
	name=sym1,
	title={符号索引}]	% 符号索引

\usepackage[unicode, bookmarksnumbered]{hyperref}	% 启动超链接和 PDF 文档信息所需
% 设置 PDF 文件信息
\hypersetup{
	pdfauthor = {郑卜凡 Bufan Zheng)},
	pdftitle = {R},
	pdfkeywords = {Algebra},
	CJKbookmarks = true}

% 用 bibLaTeX 生成参考文献
% 载入书目库: 文档类业已引入 biblatex + biber
\addbibresource{reference.bib}

\begin{document}
	\frontmatter	% 制作封面和目录.
	
	\mainmatter		% 正文开始:逐章引入 TeX 代码

	% LaTeX source for book ``代数学方法'' in Chinese
% Copyright 2018  李文威 (Wen-Wei Li).
% Permission is granted to copy, distribute and/or modify this
% document under the terms of the Creative Commons
% Attribution 4.0 International (CC BY 4.0)
% http://creativecommons.org/licenses/by/4.0/

% To be included
\chapter*{导言}	% 文档类会自动将之加入目录并设置天眉

\section*{简要说明}
\paragraph*{旨趣}
有趣的文章和科研课题都在这里了,算是一个新坑,记录一下阅读过的有趣的一些东西,和一些漫无边际的idea,最后附录存放一些微妙的问题。
\section*{致谢}
\begin{hint}
	我是真的会谢
\end{hint}
	\part{物理}
	\chapter{流体/引力对偶}
	\chapter{量子纠缠与散射振幅}
	\include{chapter3}
	\chapter{AdS/CFT对偶}
\section{大$N$极限}
本节的主要思想在\ref{B.5}中已经介绍不少了,本节目的是把目光局限在$SU(N)$ Y-M理论,这一话题的绝佳参考资料是\href{https://www.damtp.cam.ac.uk/user/tong/gaugetheory.html}{David Tong的规范场论讲义}对应章节。
\subsection{复习规范场}
首先随便\footnote{实际上这一选取不能是随便的,需要有些限制,详见Gelis\cite{Gelis:2019yfm}。}选一个规范群$G$。然后再加入一些标量场或者旋量场,它们处于$G$的某个表示之中,也就是说,在:
\begin{equation}\label{4.1.1}
	\phi_i(x)\mapsto U_{ij}\phi_j(x),\quad U_{ij}\in G
\end{equation}
的场位形变换下$\phi_i$(略去了旋量指标)的作用量应该保持不变,理论具有$G$的内禀对称性。现在我们把$U_{ij}$换成一个local的东东,也就是把每个元素都换成一个和空间位置有关的函数。显然我们利用$\phi_i$构造的拉氏量$\mathcal{L}(\partial\phi,\phi)$不一定在local的规范群下是不变的,罪魁祸首就是$\partial$,现在还要多出$\partial U$的项。可以考虑引入一个规范场$A^\mu$,在规范群的作用下如此变换:\footnote{其实就是在说其处于自伴表示下}
\begin{equation}\label{4.1.2}
	A_\mu(x)\mapsto U(x)A_\mu (x)U^\dagger (x)+\frac i g U(x)\partial_\mu U^\dagger(x)
\end{equation}
注意这里$A_\mu$是一个$N\times N$的矩阵形式的场,我们略去了矩阵指标,它也可以在生成元$T^a$基底下展开为$A^a T^a$。如果我们做下面的替换:
\begin{equation}
	\mathcal{L}(\partial\phi,\phi)\mapsto\mathcal{L}(D\phi,\phi),\quad D_\mu\equiv \partial_\mu -igA_\mu
\end{equation}
那么可以证明$\mathcal{L}$就在local规范群的作用下也是不变的,也就是说对场位形进行变换\ref{4.1.1}和\ref{4.1.2}导致的物理是不变的,而场位形本身不是可测的,我们关心的是场的激发导致的散射振幅。这意味着场位形空间是存在冗余的,这些由规范群联系起来的场位形应该看作是同一个场,路径积分也只用积一次,这直接导致了鬼场作为一个辅助场的存在。\footnote{这里有点微妙,$\phi_i$路径积分里面还是需要积分所有场位形,因为我们应该把\ref{4.1.1}看作是规范场变换\ref{4.1.2}诱导出来的一个被动的变换,所以一旦我们通过FP量子化方法取定了$A$的一个规范,那么用$U$联系的$\phi$也要看成不同。就像是现在把$A$当成横轴,$\phi$当纵轴,把$y=kx,k\in\mathbb{R}$线上的$(A,\phi)$看成等同,为了消除冗余,可以先固定$A$,那么$\phi$依然是自由的,每个$\phi$给出一个物理上独立的场位形$(A,\phi)$。}
\begin{remark}
	说到这里还是要多说一点,global的对称性和local的规范对称性看起来都是对场操作一下后$\mathcal{L}$不变,或者差个全导数。但是前者是真正的动力学对称性,也就是说可以用Noether定理推出非平凡守恒荷的,但是后者仅仅只是数学上描述的冗余,why?因为我们讲动力学对称性,话题的主角不是场位形本身,而是场的激发产生的物理态,动力学对称性就是说对态进行变换后得到一个新的态,两者的运动方程形式完全一致,是一样的动力学演化。比如中心势场问题,把物体的运动轨迹旋转一下,物理上可区分这两个不同的态,但是它们遵循相同的演化方程。而local的规范对称性只是数学上的冗余,只是拉氏量本身的对称性,没有任何的动力学意义,也就是说变换前后两个态就是物理上无法区分的,而且不会带来非平凡守恒荷。所以前者我们场位形需要全积,后者我们只需要积某个特定规范下的场位形。当然,在无穷远处不归0的规范变换实际上是真正的对称性,即所谓渐近对称性,这个就不多谈了,具体可见A.Strominger的工作。
\end{remark}

规范场有两种视角来看,第一种就是因为我们要求的散射振幅是$\left<A^aA^b\cdots\right>$这些带色指标的分量场构成的关联函数(前面还要乘上对应的极化矢量)。所以可以直接用$A^a$来写费曼规则。第二种就是由于$A^a=\tr \left(A T^a\right)$,所以我们可以先把$A^\mu$就看作一个独立的矩阵,直接从矩阵来写费曼规则,最后原则上散射振幅就能用$\left<{A^{i}}_j{A^{k}}_l\ldots\right>$求迹得到。类似于\ref{B.5}中的double line表示,对于和一个费米子耦合的YM理论:
\begin{equation}
	1
\end{equation}
有:
\begin{equation}
	1
\end{equation}
所以费曼图为:
\begin{equation}
	1
\end{equation}
我们忽略了鬼场的影响,不过就后面讨论的目的而言问题不大。
	\chapter{超对称}
\section{标准模型}
这本质上是唯象学的内容,但是超对称的提出很大程度上就是为了寻找超出标准模型的物理。标准模型从群论上看由三个规范群描述:
\begin{equation}
	\mathrm{SU}(3)_\mathrm{C}\times\mathrm{SU}(2)_\mathrm{L}\times\mathrm{U}(1)_\mathrm{Y}
\end{equation}
传递相互作用的规范玻色子在其自伴表示下,而参与构成物质的费米子处于其基本表示下,其实这一点就很不自然,我们是从为了解释实验数据而要求费米子处于基本表示,原则上来说理论允许费米子处于任意其他表示之中。

规范玻色子本身是没有质量的,但是$W^\pm/Z^0$玻色子有质量,这可以通过引入一个复标量场,Higgs,通过$\mathrm{SU}(2)_\mathrm{L}\times\mathrm{U}(1)_\mathrm{Y}\to \mathrm{U}(1)_{\mathrm{EM}}$的对称性自发破缺带来质量项,还剩下一个实标量场自由度和矢量场自由度没有破缺,它们构成Higgs粒子和光子。其它费米子质量的起源也可以通过和Higgs引入Yukawa相互作用项耦合,自发破缺后带来质量项。注意QCD中费米子quark的质量项是可以直接通过在$\mathcal{L}_{\mathrm{QCD}}$中添加正定的质量项得到,但是由于电弱规范理论是一个手征理论(注意$\mathrm{SU}(2)_\mathrm{L}$下标$L$),所以直接添加质量项会破缺手征性,质量项只能通过Higgs机制得到,详细的推导可见\href{https://yzhxxzxy.github.io/teaching/2209_SM_FR.pdf}{余钊焕老师的讲义}。也正是因为有这么个更大的群到子群的对称性自发破缺,所以电弱理论已经统一,但是QCD还单独落在外面,大统一理论(GUT)的目标就是找到更大的群如$SU(5)$,让它自发破缺到$\mathrm{SU}(2)_\mathrm{L}\times\mathrm{U}(1)_\mathrm{Y}\to \mathrm{U}(1)_{\mathrm{EM}}$,自然得到三种基本相互作用。
\subsection{标准模型拉氏量}
我们先给出一份网上流传甚广的“物理学最复杂公式”,也就是标准模型的Lagrangian,首先是QCD部分,动力学为:\footnote{选取规定\[D_\mu=\partial_\mu-\mathrm{i}g_\mathrm{s}G_\mu^at^a,\quad G^{a\mu\nu}\equiv\partial^\mu G^{a\nu}-\partial^\nu G^{a\mu}+g_\mathrm{s}f^{abc}G^{b\mu}G^{c\nu}\]}
\begin{equation}
	\begin{aligned}
		\mathcal{L}_\mathrm{QCD}=&\sum_q\bar{q}(\mathrm{i}\gamma^\mu D_\mu-m_q)q-\frac14G_{\mu\nu}^aG^{a\mu\nu},\quad q=u,d,s,c,b,t,\quad a=1,\cdots,8\\
		=&\sum_q[\bar{q}(\mathrm{i}\gamma^{\mu}\partial_{\mu}-m_{q})q+g_{\mathrm{s}}G_{\mu}^{a}\bar{q}\gamma^{\mu}t^{a}q]+\frac12[(\partial_{\mu}G_{\nu}^{a})(\partial^{\nu}G^{a\mu})-(\partial_{\mu}G_{\nu}^{a})(\partial^{\mu}G^{a\nu})]\\
		&-g_{\mathrm{s}}f^{abc}(\partial_{\mu}G_{\nu}^{a})G^{b\mu}G^{c\nu}-\frac{1}{4}g_{\mathrm{s}}^{2}f^{abc}f^{ade}G_{\mu}^{b}G_{\nu}^{c}G^{d\mu}G^{e\nu}.
	\end{aligned}
\end{equation}
然后需要引入FP鬼场量子化方法固定规范,还要加入两项:
\begin{equation}
	\mathcal{L}_{\mathrm{QCD,GF}}=-\frac1{2\xi}(\partial^{\mu}G_{\mu}^{a})^{2}
\end{equation}
和鬼场:
\begin{equation}
	\mathcal{L}_{\mathrm{QCD,FP}}=-\bar{\eta}_{g}^{a}\left(g_{\mathrm{s}}\frac{\delta G^{a}}{\delta\alpha^{c}}\right)\eta_{g}^{c}=-\bar{\eta}_{g}^{a}(\delta^{ac}\partial^{2}+g_{\mathrm{s}}f^{abc}\partial^{\mu}G_{\mu}^{b})\eta_{g}^{c}\rightarrow-\bar{\eta}_{g}^{a}\delta^{ab}\partial^{2}\eta_{g}^{a}+g_{\mathrm{s}}f^{abc}(\partial^{\mu}\eta_{g}^{a})G_{\mu}^{b}\eta_{g}^{c}.
\end{equation}
这里鬼场是Grassmannian。电弱规范理论就麻烦很多,自发破缺后得到真正和粒子对应的场要由自发破缺前的场通过适当的线性组合得到,这里只给最终结论。首先是希格斯机制带来的规范玻色子质量项
\begin{equation}
	\mathcal{L}_\mathrm{GBM}=m_W^2W^{+\mu}W_\mu^-+\frac12m_Z^2Z^\mu Z_\mu 
\end{equation}
然后是Yukawa耦合带来的Higgs粒子和费米子质量项:
\begin{equation}
	\mathcal{L}_{\mathrm{Y}}=-m_{d_i}\bar{d}_id_i-m_{u_i}\bar{u}_iu_i-m_{\ell_i}\bar{\ell}_i\ell_i-\frac{m_{d_i}}vH\bar{d}_id_i-\frac{m_{u_i}}vH\bar{u}_iu_i-\frac{m_{\ell_i}}vH\bar{\ell}_i\ell_i
\end{equation}
这里$i$隐含对三代quark求和。费米子和规范场之间通过协变导数项引入相互作用,或者说等价于和一堆$U(1)$的流耦合:
\begin{equation}
	\mathcal{L}_{\mathrm{EWF}}\supset A_{\mu}J_{\mathrm{EM}}^{\mu}+Z_{\mu}J_{Z}^{\mu}+W_{\mu}^{+}J_{W}^{+,\mu}+W_{\mu}^{-}J_{W}^{-,\mu}
\end{equation}
其中:
\begin{equation}
	\begin{gathered}
		J_\mathrm{EM}^\mu\equiv\sum_fQ_fe\bar{f}\gamma^\mu f\\
		\begin{aligned}J_Z^\mu&\equiv\frac g{2c_\mathrm{W}}\sum_f\bar{f}\gamma^\mu(g_\mathrm{V}^f-g_\mathrm{A}^f\gamma^5)f=\frac g{c_\mathrm{W}}\sum_f(g_\mathrm{L}^f\bar{f}_\mathrm{L}\gamma^\mu f_\mathrm{L}+g_\mathrm{R}^f\bar{f}_\mathrm{R}\gamma^\mu f_\mathrm{R})\end{aligned}\\
		J_{W}^{+,\mu}\equiv\frac g{\sqrt{2}}(\bar{u}_{i\mathrm{L}}\gamma^{\mu}V_{ij}d_{j\mathrm{L}}+\bar{\nu}_{i\mathrm{L}}\gamma^{\mu}\ell_{i\mathrm{L}}),\quad J_{W}^{-\mu}\equiv(J_{W}^{+\mu})^{\dagger}=\frac g{\sqrt{2}}(\bar{d}_{j\mathrm{L}}V_{ji}^{\dagger}\gamma^{\mu}u_{i\mathrm{L}}+\bar{\ell}_{i\mathrm{L}}\gamma^{\mu}\nu_{i\mathrm{L}})
	\end{gathered}
\end{equation}
这里$f$表示任意的费米子,$\ell$表示轻子,$u,d$表示夸克,$\nu$是中微子。$SU(2)\times U(1)$的规范场自相互作用可以由动能项:
\begin{equation}
	\mathcal{L}_\mathrm{EWG}=-\frac14W_{\mu\nu}^aW^{a\mu\nu}-\frac14B_{\mu\nu}B^{\mu\nu},
\end{equation}
来导出,注意这里的$W^a,B$都是未破缺之前的场,破缺后的场需要线性组合得到$W^{\pm}/Z^0,A^\mu$,经过贼复杂的计算后得到:
\begin{equation}
	\begin{aligned}
		\mathcal{L}_{\mathrm{EWG}}= & \frac{1}{2}\left[\left(\partial_{\mu} A_{\nu}\right)\left(\partial^{\nu} A^{\mu}\right)-\left(\partial_{\mu} A_{\nu}\right)\left(\partial^{\mu} A^{\nu}\right)\right]+\frac{1}{2}\left[\left(\partial_{\mu} Z_{\nu}\right)\left(\partial^{\nu} Z^{\mu}\right)-\left(\partial_{\mu} Z_{\nu}\right)\left(\partial^{\mu} Z^{\nu}\right)\right] \\
		& +\left(\partial_{\mu} W_{\nu}^{+}\right)\left(\partial^{\nu} W^{-\mu}\right)-\left(\partial_{\mu} W_{\nu}^{+}\right)\left(\partial^{\mu} W^{-\nu}\right)+\frac{g^{2}}{2}\left(W_{\mu}^{+} W^{+\mu} W_{\nu}^{-} W^{-\nu}-W_{\mu}^{+} W^{+\nu} W_{\nu}^{-} W^{-\mu}\right) \\
		& +\mathrm{i} e\left[\left(\partial_{\mu} W_{\nu}^{+}\right) W^{-\mu} A^{\nu}-\left(\partial_{\mu} W_{\nu}^{+}\right) W^{-\nu} A^{\mu}-W^{+\mu}\left(\partial_{\mu} W_{\nu}^{-}\right) A^{\nu}+W^{+\nu}\left(\partial_{\mu} W_{\nu}^{-}\right) A^{\mu}\right. \\
		& \left.\quad+W^{+\mu} W^{-\nu}\left(\partial_{\mu} A_{\nu}\right)-W^{+\nu} W^{-\mu}\left(\partial_{\mu} A_{\nu}\right)\right] \\
		& +\mathrm{i} g c_{\mathrm{W}}\left[\left(\partial_{\mu} W_{\nu}^{+}\right) W^{-\mu} Z^{\nu}-\left(\partial_{\mu} W_{\nu}^{+}\right) W^{-\nu} Z^{\mu}-W^{+\mu}\left(\partial_{\mu} W_{\nu}^{-}\right) Z^{\nu}+W^{+\nu}\left(\partial_{\mu} W_{\nu}^{-}\right) Z^{\mu}\right. \\
		& \left.\quad+W^{+\mu} W^{-\nu}\left(\partial_{\mu} Z_{\nu}\right)-W^{+\nu} W^{-\mu}\left(\partial_{\mu} Z_{\nu}\right)\right] \\
		& +e^{2}\left(W_{\mu}^{+} W^{-\nu} A_{\nu} A^{\mu}-W_{\mu}^{+} W^{-\mu} A_{\nu} A^{\nu}\right)+g^{2} c_{\mathrm{W}}^{2}\left(W_{\mu}^{+} W^{-\nu} Z_{\nu} Z^{\mu}-W_{\mu}^{+} W^{-\mu} Z_{\nu} Z^{\nu}\right)\\
		&+egc_{\mathrm{W}}(W_{\mu}^{+}W^{-\nu}A_{\nu}Z^{\mu}+W_{\mu}^{+}W^{-\nu}A^{\mu}Z_{\nu}-2W_{\mu}^{+}W^{-\mu}A_{\nu}Z^{\nu})
	\end{aligned}
\end{equation}

把前面讨论的这一堆拉氏量全部凑在一堆就是标准模型了!但实际计算上我们不会真的取考虑全部的标准模型,往往是对一部分模型积掉实验能标上更高能的自由度得到一个有效理论,比如$\pi$介子理论,但是也够难算的了。
\subsection{标准模型的局限}
虽然标准模型在解释实验现象上取得了巨大的成功,但是仍有非常多的问题亟待解决,关于唯象上新物理的寻找的更多内容可见书籍\cite{baer_weak_2006}的前两章。
\begin{description}
	\item[参数过多] 描述
	\item[没有引力]
	\item[中微子震荡]
\end{description}
	\part{数学}
	\include{chapter1-math}
	% 如有附录则在以下引入
	\part{附录}
	 \appendix
	 \chapter{物理疑难杂症}
\section{真空态与绘景}
	 \chapter{数学疑难杂症}
\section{直积$\cdot$张量积$\cdot$直和}
物理人在这些概念上往往非常模糊,胡乱使用,现在我们使用物理人的思想来区分下这几个概念。由于这几个概念的使用场景是在量子力学,所以我们在向量空间上讨论这三个运算。

直积和张量积是紧密相连的,这两个概念一起介绍。直积从定义上讲就是给两个集合,然后把两个集合简单的并在一起构成一个更大的集合,$A\times B$,仅此而已。但我们一般会在上面进一步定义内积和加法数乘使得其成为一个线性空间:
\begin{description}
	\item[加法] $(a,b)+(a^\prime,b^\prime)=(a+a^\prime,b+b^\prime)$
	\item[数乘] $\lambda (a,b)=(\lambda a,\lambda b),\quad\lambda\in\mathbb{F}$
	\item[内积] $(a,b)\cdot (a^\prime,b^\prime)= a\cdot a^\prime +b\cdot b^\prime$
\end{description}
这样构成的空间称为$\mathbb{F}$上的自由向量空间:
\begin{equation}
	\mathcal{F}(V,W;\mathbb{F})\equiv \left\{\sum_{(v,w)\in V\times W}k_{v,w}(v,w),k_{v,w}\in\mathbb{F}\right\}
\end{equation}
这个向量空间非常大,就是把每个$(u,v)\in V\times W$都拿来作为基底张成的线性空间。张量积的初衷是想去找$V\times W\to Z$上的双线性函数$f$,双线性函数一定满足下面的条件:
\begin{equation}
	\begin{aligned}&\text{ (}k_1v_1+k_2v_2,w)\sim k_1(v_1,w)+k_2(v_2,w)\\&(v,k_1w_1+k_2w_2)\sim k_1(v,w_1)+k_2(v,w_2)\end{aligned}
\end{equation}
$\sim$表示它们作用$f$得到的值一样。这么来看原先的那个$V\times W$还是太大了,无法自然地蕴含上面的等价关系,所以我们干脆把上面的两条等价关系给模掉,得到一个更合适的线性空间$Y$:
\begin{equation}
	Y\equiv \mathcal{F}(V,W)/\sim
\end{equation}
数学人更喜欢用的不是上面两条,而是和它们等价的下面四条:
\begin{equation}
	\begin{aligned}
		&\begin{aligned}(v_1+v_2,w)\sim(v_1,w)+(v_2,w)\end{aligned} \\
		&\begin{aligned}(v,w_1+w_2)\sim(v,w_1)+(v,w_2)\end{aligned} \\
		&(kv,w)\sim k(v,w) \\
		&(v,kw)\sim k(v,w)
	\end{aligned}
\end{equation}

从$V\times W$到$Y$的线性映射我们记为$h$,则$f$就自然诱导出来了$Y\to Z$的线性映射$g$,而且$Y$也小多了,我们也没必要去强调$g$的双线性性质,现在$Y$自己就蕴含了线性映射的双线性性。我们把这个新的空间叫做$V\otimes W$,即张量积空间。物理上我们把里面的元素写为$\ket{\psi}\ket{\phi}$,而且前面的四条性质就蕴含在我们物理上对张量积的普遍共识,物理上对于张量积的应用一般是体系有多个自由度,比如多个粒子或者一个粒子但是有自旋这种自由度,那么整个希尔伯特空间就看作是每个自由度的希尔伯特空间的张量积。而且在每个自由度上的矢量加法满足$\otimes$的分配律。

由于任何一个映射$f$都可以诱导出映射$g$,所以很多时候我们不会额外区分两者,事实上可以证明双线性映射空间$\mathscr{L}(V,W;Z)$和$\mathscr{L}(V\otimes W;Z)$是同构的。在物理上我们考虑的线性空间都是内积空间\footnote{尽管张量积的定义不需要内积,对偶空间也不需要,这里我们只看有内积的简单情况。},根据里斯表示定理,每个向量空间中的元素都可以和其对偶空间中的元素通过内积双线性性建立一一对应,也即所谓ket和bra的概念。如果取:
\begin{equation}
	(\bra{\psi_1}\otimes\bra{\phi_1})\cdot (\ket{\phi_1}\otimes\ket{\psi_2})=\braket{\phi_1}{\phi_2}\cdot\braket{\psi_1}{\psi_2}
\end{equation}
那么很容易说明$V^*\otimes W^*=(V\otimes W)^*$。当然,这一点的成立并不依赖于上面的内积选取\footnote{毕竟它的定义就不需要内积,这里局限在内积空间上讨论感觉从物理直观上更容易说清楚,用严谨的general的数学反而迷糊。},只是量子力学里面都是这么取的。

上面的定义是构造性的,但数学上更喜欢的是泛性质的定义,直接用下面的交换图就好了:
\begin{definition}
	张量积空间是某个向量空间$Y$ 配以双线性映射 $h:V\times W\to Y$ ,使得对于任意双线性映射 $f:V\times W\to Z$ ,存在唯一的线性映射 $g:Y\to Z$ ,使 $f=g\circ h$ 。
	\begin{center}
		\begin{tikzcd}
			V\times W \arrow[r,"h"] \arrow[dr,"f"]
			& Y= V\otimes W\arrow[d,dashrightarrow,"g"]\\
			& Z
		\end{tikzcd}
	\end{center}
	上面交换图中虚线的意思是存在且唯一存在一个映射使得图标交换。
\end{definition}

直和和直积在数学上真的不怎么区分,我先给出泛性质的定义你就知道它们之间的区别有多么微妙了。


从范畴的角度看它们差的仅仅只是一个是嵌入一个是投影,把箭头反过来罢了!所以数学上真的不怎么区分,但是这里说的是外直和,后面还会讲到内直和。
	 \backmatter
	% 使用 bibLaTeX 制作书目
	\printbibliography[heading=bibintoc, title=参考文献]


	% 制作索引: 先是符号索引, 继而是名词索引暨英译
%	{\footnotesize
%	\printindex[sym1]
%	\indexprologue{中文术语按汉语拼音排序.}
%	\printindex
%		% 如有需要, 加入表格和图片索引
%		\cleardoublepage
%	 	\phantomsection
%		\addcontentsline{toc}{chapter}{\listfigurename}
%		\listoffigures
%		\cleardoublepage
%	 	\phantomsection
%		\addcontentsline{toc}{chapter}{\listtablename}
%		\listoftables
%	}
\end{document}