\chapter{量子力学与Morse理论}
Morse理论是几何上的重要定理,微分几何中对一个流形我们可以考虑上面的标量场,其微分的定义也是十分自然的。如果在某点$p\in\mathcal{M}$,$(df)|_p=0$,而且这样的点都是孤立点且Hessian矩阵非退化,那么就称这样的函数是Morse函数。而Morse定理是一组不等式,告诉我们这些微分为0的点的个数看似是流形局部的性质,其实是由流形的整体拓扑决定的。后面会详细介绍这一理论,我们先考虑一个最简单的例子。考虑一个可定向二维闭曲面$gT^2$,Hessian矩阵不退化,那么可以根据Hessian矩阵正负本征值个数把函数微分为0的点分类为极大值、极小值和鞍点。Morse定理告诉我们:
\begin{equation}
	\begin{cases}
		\text{极大值点的个数}&\geq 1\\
		\text{极小值点的个数}&\geq 1\\
		\text{鞍点的个数}&\geq 2g
	\end{cases}
\end{equation}
前面两个不等式其实是在说流形的连通分支个数是1,也是流形整体的性质。这个定理显然是可以从纯数学的角度来证明的,但是Witten在1982年对这一理论给出了(超对称)量子力学版本的证明\cite{Witten:1982im},这一角度十分之有趣,后面讲会从比较偏重于数学的角度来考虑这一证明。
\section{量子力学基础}
本节的目的差不多是量子力学补遗,看下数学家思考量子力学的角度。
\subsection{Wyle算符}
量子力学里面所谓量子化就是把力学量从函数变成算符:
\begin{equation}
	\{f,g\}\Rightarrow\{\hat f,\hat g\}_q=\frac{1}{i\hbar }[\hat f,\hat g]
\end{equation}
这里$\{\cdot\}$是经典Poisson括号,下标$q$表示对应量子化后的量子泊松括号,它和对易子成正比。当我们用哈密顿形式,把动力学方程完全用Possion括号的形式写出来,然后全部做了这样的替换之后量子化也就完成了。而量子力学基本假设里面很重要的一条是量子化之后位置和动量算符不对易,这导致在涉及到$f(x,p)=x^mp^n$的力学量量子化时会出现定义模糊。实际上这一问题解决有很多种方案,我们采取从数学角度上讲比较好的方法。

在位置表象下考虑,这时$\hat x\to x,\hat p\to-i\hbar\partial_x$,考虑力学量$f(x,p)=\sum_m f_m(x)p^m$,尝试下面两种极端的量子化方案,作用在任意一个函数$v(x)$上为:
\begin{equation}
	\begin{aligned}
		&f^+v =\sum_m f_m(x)\left(-i\hbar\partial_x\right)^m v(x)=\sum_m \left.\left(-i\hbar\partial_x\right)^m f_m(y)v(x)\right|_{y=x}\\
		&f^-v =\sum_m \left(-i\hbar\partial_x\right)^m \left[f_m(x)v(x)\right]=\sum_m \left.\left(-i\hbar\partial_x\right)^m f_m(x)v(x)\right|_{y=x}
	\end{aligned}
\end{equation}
Wyle算符就定义称折中的方案:
\begin{equation}
	\hat f_{\text{Wyle}} v \equiv \sum_m \left.\left(-i\hbar\partial_x\right)^m f_m\left(\frac{x+y}{2}\right)v(x)\right|_{y=x}
\end{equation}
这样做量子化会让半经典极限,也就是$\hbar\to 0$的极限性质比较好。

\begin{theorem}[半经典极限]
	$\hbar \to 0$,有:
	\begin{equation}
		\hat f \hat g=\widehat{fg}+\mathcal{O}(\hbar),\quad\left\{\hat f,\hat g\right\}_q=\widehat{\left\{f,g\right\}}+\mathcal{O}(\hbar)
	\end{equation}
\end{theorem}
\begin{proof}
 	使用数学归纳法进行证明,
\end{proof}

\subsection{一维散射问题}
考虑一个在$[x_1,x_2]$内才不为0的势场,入射波能量$E>0$。量子力学对这个问题已经充分的研究过了,我们只需在平面波边界条件下求解下面的薛定谔方程就好:
\begin{equation}\ref{7.6}
	-\frac{\hbar^2}{2}\psi^{\prime\prime}+V(x)\psi= E\psi
\end{equation}
其中我们已经选取合适的量纲使得$m=1$。数学上并不是解方程这么玩的,下面我们就来欣赏一下数学家的做法,数学家或许不能告诉你这个微分方程具体的解,但是他却能告诉你非常多的解的性质。首先假设方程\ref{7.6}的解空间为$L$,这是一个二维的函数线性空间。现在考虑$(-\infty,x_1)\cup(x_2,+\infty)$上的薛定谔方程:
\begin{equation}\ref{7.7}
	-\frac{\hbar^2}{2}\psi^{\prime\prime}= E\psi
\end{equation}
对应的解空间设为$L_0=\operatorname{span}\{e_1=\sin(kx),e_2=\cos(kx)\}$。则可以定义单值化算子:
\begin{definition}[单值化算子]
	$B_{\pm}:L\to L_0,u(x)\in L,u_0(x)\in L_0$定义:
	\begin{equation}
		\begin{aligned}
			&B_- u (x)=u_0(x),\quad \left.u=u_0\right|_{x<x_1}\\
			&B_+ u (x)=u_0(x),\quad \left.u=u_0\right|_{x>x_2}
		\end{aligned}
	\end{equation}
	根据ode的解唯一存在性定理,可知$B_{\pm}$是同构,下面的单值化算子是well-define的:
	\begin{equation}
		M\equiv B_+B_-^{-1}
	\end{equation}
\end{definition}
熟悉高等量子散射理论的我们立刻就可以看出这里的$B_{\pm}$其实就是Moller算符$\Omega_{\pm}$,而这里的$M$其实就是散射矩阵$S\equiv\Omega_-^\dagger\Omega_+$。

在$L_0$中选取正弦余弦作为基底,设$\eta\xi\in L_0$,定义下面的楔积,用对易子符号表示\footnote{有点符号混用}:
\begin{equation}
	\left[\xi,\eta\right]\equiv \xi\wedge \eta =\xi_1\eta_2-\xi_2\eta_1
\end{equation}
\begin{theorem}\label{7.1.3}
	$M$保$\left[\cdot,\cdot\right]$不变。$\left[M(\xi),M(\eta)\right]=\left[\xi,\eta\right]$
\end{theorem}
为了证明这个定理我们先证个引理,其中遇到的概念后面也会非常有用。
\begin{definition}[斜积]
	\begin{equation}
		\left\{\psi,\phi\right\}\equiv\psi^\prime\phi-\psi\phi^\prime
	\end{equation}
\end{definition}
这其实就是朗斯基行列式的定义,而且不难证明其不依赖于$x$,是一个常数,也即$\partial_x\left\{\psi,\phi\right\}=0$。
\begin{lemma}
	\begin{equation}
		\{\psi,\phi\}=-k\left[B_\pm\psi,B_\pm\phi\right]
	\end{equation}
\end{lemma}
\begin{proof}
	\begin{equation}
		\begin{aligned}
			1
		\end{aligned}
	\end{equation}
\end{proof}
现在来证明定理\ref{7.1.3}:
\begin{proof}
	
\end{proof}
\begin{corollary}
	$\det M=1$
\end{corollary}

注意我们到现在为止讨论的都是实值函数空间,实际上量子力学里面更方便的做法是考虑$L_0,L_1$的复化,这样做会方便很多。
\begin{definition}[复化]
	
\end{definition}
\begin{remark}
	这其实涉及到量子力学为什么需要复数,能否构造纯粹在实数空间上定义的量子力学?近年来潘建伟老师组的实验工作否定了这一点\cite{Renou:2021dvp,Chen:2021ril},我从文章\cite{karam_why_2020}的角度来稍微提一下。
\end{remark}