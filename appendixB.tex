\chapter{数学疑难杂症}
\section{直积$\cdot$张量积$\cdot$直和}
物理人在这些概念上往往非常模糊,胡乱使用,现在我们使用物理人的思想来区分下这几个概念。由于这几个概念的使用场景是在量子力学,所以我们在向量空间上讨论这三个运算。

直积和张量积是紧密相连的,这两个概念一起介绍。直积从定义上讲就是给两个集合,然后把两个集合简单的并在一起构成一个更大的集合,$A\times B$,仅此而已。但我们一般会在上面进一步定义内积和加法数乘使得其成为一个线性空间:
\begin{description}
	\item[加法] $(a,b)+(a^\prime,b^\prime)=(a+a^\prime,b+b^\prime)$
	\item[数乘] $\lambda (a,b)=(\lambda a,\lambda b),\quad\lambda\in\mathbb{F}$
	\item[内积] $(a,b)\cdot (a^\prime,b^\prime)= a\cdot a^\prime +b\cdot b^\prime$
\end{description}
这样构成的空间称为$\mathbb{F}$上的自由向量空间:
\begin{equation}
	\mathcal{F}(V,W;\mathbb{F})\equiv \left\{\sum_{(v,w)\in V\times W}k_{v,w}(v,w),k_{v,w}\in\mathbb{F}\right\}
\end{equation}
这个向量空间非常大,就是把每个$(u,v)\in V\times W$都拿来作为基底张成的线性空间。张量积的初衷是想去找$V\times W\to Z$上的双线性函数$f$,双线性函数一定满足下面的条件:
\begin{equation}
	\begin{aligned}&\text{ (}k_1v_1+k_2v_2,w)\sim k_1(v_1,w)+k_2(v_2,w)\\&(v,k_1w_1+k_2w_2)\sim k_1(v,w_1)+k_2(v,w_2)\end{aligned}
\end{equation}
$\sim$表示它们作用$f$得到的值一样。这么来看原先的那个$V\times W$还是太大了,无法自然地蕴含上面的等价关系,所以我们干脆把上面的两条等价关系给模掉,得到一个更合适的线性空间$Y$:
\begin{equation}
	Y\equiv \mathcal{F}(V,W)/\sim
\end{equation}
数学人更喜欢用的不是上面两条,而是和它们等价的下面四条:
\begin{equation}
	\begin{aligned}
		&\begin{aligned}(v_1+v_2,w)\sim(v_1,w)+(v_2,w)\end{aligned} \\
		&\begin{aligned}(v,w_1+w_2)\sim(v,w_1)+(v,w_2)\end{aligned} \\
		&(kv,w)\sim k(v,w) \\
		&(v,kw)\sim k(v,w)
	\end{aligned}
\end{equation}

从$V\times W$到$Y$的线性映射我们记为$h$,则$f$就自然诱导出来了$Y\to Z$的线性映射$g$,而且$Y$也小多了,我们也没必要去强调$g$的双线性性质,现在$Y$自己就蕴含了线性映射的双线性性。我们把这个新的空间叫做$V\otimes W$,即张量积空间。物理上我们把里面的元素写为$\ket{\psi}\ket{\phi}$,而且前面的四条性质就蕴含在我们物理上对张量积的普遍共识,物理上对于张量积的应用一般是体系有多个自由度,比如多个粒子或者一个粒子但是有自旋这种自由度,那么整个希尔伯特空间就看作是每个自由度的希尔伯特空间的张量积。而且在每个自由度上的矢量加法满足$\otimes$的分配律。

由于任何一个映射$f$都可以诱导出映射$g$,所以很多时候我们不会额外区分两者,事实上可以证明双线性映射空间$\mathscr{L}(V,W;Z)$和$\mathscr{L}(V\otimes W;Z)$是同构的。在物理上我们考虑的线性空间都是内积空间\footnote{尽管张量积的定义不需要内积,对偶空间也不需要,这里我们只看有内积的简单情况。},根据里斯表示定理,每个向量空间中的元素都可以和其对偶空间中的元素通过内积双线性性建立一一对应,也即所谓ket和bra的概念。如果取:
\begin{equation}
	(\bra{\psi_1}\otimes\bra{\phi_1})\cdot (\ket{\phi_1}\otimes\ket{\psi_2})=\braket{\phi_1}{\phi_2}\cdot\braket{\psi_1}{\psi_2}
\end{equation}
那么很容易说明$V^*\otimes W^*=(V\otimes W)^*$。当然,这一点的成立并不依赖于上面的内积选取\footnote{毕竟它的定义就不需要内积,这里局限在内积空间上讨论感觉从物理直观上更容易说清楚,用严谨的general的数学反而迷糊。},只是量子力学里面都是这么取的。

上面的定义是构造性的,但数学上更喜欢的是泛性质的定义,直接用下面的交换图就好了:
\begin{definition}
	张量积空间是某个向量空间$Y$ 配以双线性映射 $h:V\times W\to Y$ ,使得对于任意双线性映射 $f:V\times W\to Z$ ,存在唯一的线性映射 $g:Y\to Z$ ,使 $f=g\circ h$ 。
	\begin{center}
		\begin{tikzcd}
			V\times W \arrow[r,"h"] \arrow[dr,"f"]
			& Y= V\otimes W\arrow[d,dashrightarrow,"\exists !g"]\\
			& Z
		\end{tikzcd}
	\end{center}
	上面交换图中虚线的意思是存在且唯一存在一个映射使得图标交换。
\end{definition}

直和和直积在数学上真的不怎么区分,我先给出泛性质的定义你就知道它们之间的区别有多么微妙了。
\begin{definition}
	直积在数学上指的是\textbf{积范畴},即对于范畴 $\mathcal{C}$ 中的一族对象 $X_i\left(i\in I\right)$,若存在一个对象,通常记为 $\prod_{i\in I}X_i$ ,以及一族态射$\pi_j:\prod_{i\in I}X_i\to X_j$使得对任意对象 $X$ 和一族态射$f_j:X\to X_j$,都存在唯一的态射$f:X\to\prod_{i\in I}X_i$使得对任意 $i\in I$ 都有$	\pi_i\circ f=f_i$
	那么对象 $\prod_{i\in I}X_i$ 连同态射族 $\{\pi_i:i\in I\}$ 称为对象族 $\{X_i:i\in I\}$ 的积(product)。画出交换图如下:
	\begin{center}
		\begin{tikzcd}
			X \arrow[r,dashrightarrow,"\exists ! f"] \arrow[dr,"f_j" swap]
			& \prod_{i\in I} X_i\arrow[d,"\pi_j"]\\
			& X_j
		\end{tikzcd}
	\end{center}
\end{definition}

\begin{definition}
	直和在数学上指的是\textbf{余积范畴}(coproduct),交换图上的定义只需要把上面图的箭头全部反转就好了,文字定义不难相应写出:
	\begin{center}
		\begin{tikzcd}
			X \arrow[r,dashleftarrow,"\exists ! f"]\arrow[dr,leftarrow,"f_j",swap]
			& \coprod_{i\in I} X_i\arrow[d,leftarrow,"\iota_j"]\\
			& X_j 
		\end{tikzcd}
	\end{center}
	可见两者定义上差的只是一个是投影$\pi$,一个是嵌入$\iota_i$。
\end{definition}
从范畴的角度看它们差的仅仅只是一个是积一个是余积,把箭头反过来罢了!所以数学上真的不怎么区分,特别是对于线性空间范畴,这个定义真的就是等价的!从维数上也能看出这一点$\mathbb{R}^m\times\mathbb{R}^n=\mathbb{R}^m\oplus\mathbb{R}^n=\mathbb{R}^{m+n}$,但是前面的张量积$\mathbb{R}^{m}\otimes\mathbb{R}^{n}=\mathbb{R}^{mn}$。但是这里说的是外直和,后面还会讲到内直和。

回到物理一些的想法,翻阅喀兴林老师的高量课本,第一章就给出来直和的三条性质:
\begin{description}
	\item[数乘分配]$\lambda \left(\ket{\psi}\oplus\ket{\phi}\right)=\lambda\ket{\psi}\oplus\lambda\ket{\phi}$
	\item[加法] $\ket{\psi}\oplus\ket{\phi}+\ket{\psi^\prime}\oplus\ket{\phi^\prime}=\left(\ket{\psi}+\ket{\psi^\prime}\right)\oplus\left(\ket{\phi}+\ket{\phi^\prime}\right)$
	\item[内积]$\left(\ket{\psi}\oplus\ket{\phi},\ket{\psi^\prime}\oplus\ket{\phi^\prime}\right)=\braket{\psi}{\phi}+\braket{\psi^\prime}{\phi^\prime}$
\end{description}
这不和自由向量空间的定义一模一样么!所以在线性空间范畴上真看不出啥区别。我们再看\textbf{直和}和一般的\textbf{向量的和}有啥区别。注意我们定义主打的就是两个希尔伯特空间是互相分开的,互相不影响,我们在定义两个向量空间的外直和时,我们并不要求两个向量空间的元素之间有加法定义,而是定义直和来把多个自由度强行直接并起来。这其实有某种选择定则在里面,当我们把两个态加起来的时候,我们认为它们处于同一个线性空间,可以叠加得到混合态。但是你绝对不会把一个自旋为0和一个自旋为$\frac12$的态直接叠加起来,因为这意味着一个单粒子态一会儿是费米子,一会儿是玻色子,尽管理论上这种禁闭没有严格证明(见Weinberg场论第一卷相关部分),但是我们从实验上可以相信这种禁闭是存在的,而这个时候考虑总的体系希尔伯特空间,最好的方式就是用直和了,也就是直接直积并起来罢了。

量子力学里面更多的去讲直和是讲巨正则系综构造Fock空间。由于非相对论性的量子力学是个粒子数不变的体系,所以也存在上面与自旋类似的选择定则,也就是说你不会把两个粒子数不同的态直接加起来构造一个混合态,这是没有物理意义的。要用的时候就直接把他们看作是处于完全不同的希尔伯特空间,有自己的一套加法数乘规则,而且互不影响,最后再把它们通过直和直接并起来,构成一个系综,是一个混合态。

内直和就完全不一样了,内直和我们首先是有一个很大的向量空间,然后想去考虑他的分解,但是我们比较关注“正交分解”,也就是说希望把$V$分解为一系列子空间$V_1,V_2,\cdots$,这些子空间互相除了零元素没有公共元素,但它们之间的元素是有子空间自然诱导的加法的。如果对于$V$中的任意一个元素$v$,都存在且唯一一组$(v_1,v_2,\cdots)$使得$v=\sum_i v_i$,则称$V$是$V_1,V_2,\ldots$的(内)直和,也记为:
\begin{equation}
	V=\bigoplus_{i=1} V_i
\end{equation}

在讨论群表示的时候涉及到群表示的分解,那个分解实际上就是在把群表示空间分解为一系列更小的空间的内直和,每个最小的不能再分的块就是不可约表示。在纯粹考虑矩阵理论的时候也会谈及张量积和直和的概念,矩阵的直和应该对应到内直和,而张量积就是前面讨论的张量积(把矩阵看成是线性算符,$V\otimes W$看成是$(V^*\otimes W^*)^*$),只是把前面的理论用线性算符重写了一遍,并选取了一组基底写成矩阵形式。不少文献也把矩阵的张量积叫克罗内克积。

\section{Grassmann几何}

\section{李超代数}
物理人群环域模体格这些代数结构真正关心的就是群,而且很多情况下只关心它的表示,为了更好地数学上刻画李超代数,我们从环和域开始讲起。

回忆群的定义,是在集合上定义了一个乘法结构,而且封闭、结合、存在幺元且元素都可逆,现在考虑把这四条性质一条条加上去。如果只有封闭性,则称为\textbf{原群(Magama)},考虑继续加性质:
\begin{definition}[半群与幺半群]
	带有二元运算的非空集 $S$ 若满足结合律,则称之为\textbf{半群}.存在幺元的半群称为\textbf{幺半群(Monoid)}.若$M$ 是幺半群,而子集$M^{\prime}\subset M$ 满足 (i)$M^{\prime\prime}$ 对乘法封闭,(ii) $1\in M^{\prime}$,则称 $M^{\prime}$ 为 $M$ 的子幺半群.
\end{definition}
如果在幺半群的基础上加上所有元素都可逆,那就得到了群的定义。
\begin{definition}
	(含幺)环是一组资料 $(R,+,\cdot)$,其中
	\begin{itemize}
		\item [1.]$(R,+)$ 是交换群,二元运算用加法符号记作 $(a,b)\mapsto a+b$, 加法幺元记为 0, 称之为 $R$ 的加法群;
		\item [2.]乘法运算 $\cdot:R\times R\to R$ 简记为 $a\cdot b=ab$, 满足下述性质:对所有 $a,b,c\in R$, $\diamond a( b+ c) = ab+ ac, \quad ( b+ c) a= ba+ ca\quad ( $分配律,或曰双线性$) , $
		$\diamond\:a(bc)=(ab)c\quad($乘法结合律);
		\item[3.]存在元素 1$\in R$ 使得对所有 $a\in R$ 皆有 $a\cdot1=a=1\cdot a$, 称作 $R$ 的 (乘法) 幺元.
	\end{itemize}
		后面两点其实蕴含了$(R,\cdot)$是一个幺半群。除去和幺元相关性质得到的$(R,+,\cdot)$ 称作无幺环. 若子集 $S\subset R$ 对 $(+,\cdot)$ 也构成环, 并且和 $R$ 共用同样的乘法幺元 1,则称 $S$ 为 $R$ 的子环,或称 $R$ 是 S 的环扩张或扩环.
\end{definition}
\begin{example}
	后面用的比较多的是$\mathbb{Z}_n$环,它的元素由整数关于$n$的模得到,比如$\mathbb{Z}_2=\{\bar 0,\bar 1\}$,加法和乘法就是自然数的加法乘法取模$n$。全体整数也构成一个环,称为自由环$\mathbb{Z}$。
\end{example}
\begin{definition}
	域是一组资料$(K,+,\cdot)$,其中
	\begin{itemize}
		\item [1.]$(R,+)$ 是交换群;
		\item [2.]$(R,\cdot)$ 是交换群;
		\item [3.]$\forall a,b,c\in K,a\cdot (b+c)=a\dot b+a\dot c$.
	\end{itemize}
	这个定义和环颇有些相似,实际上域可以看作是环上的乘法幺半群附加交换性,这时候环变成一个\textbf{交换环},再要求乘法意义上环中除了加法零元每个元素都有逆,这环变成\textbf{除环(体)}。\textbf{域就是一个交换可除环}。
\end{definition}
简单点说就是一个集合上面定义了加减乘除,满足通常的结合律、交换律和分配律等等。但是有一个地方和小学学的数域有点不一样,若干个相同的不为0的数求和可以为0。比如$\mathbb{Z}_2$也可以看作一个域,其中$\bar 1+\bar 1=\bar 0$!
\begin{definition}
	一个域只可能出现两种情况:
	\begin{description}
		\item[特征$0$:] 任何元自己和自己加法无论多少次都不会得到零元(零元自己除外);
		\item[特征$p$:] $\exists p\in\mathbf{Prime}$满足任何数自己和自己相加$p$次后总是得到零元。
 	\end{description}
\end{definition}
后面的讨论在特征为$0$的域上进行。

回忆李代数首先是一个代数,生活在一个线性空间,李超代数首先是一个超代数,和代数的区别是其生活在阶化之后的线性空间上。
\begin{definition}[$Z_n$阶化向量空间]
	若一个向量空间可以直和分解为$V=\bigoplus_{\alpha\in\mathbb{Z}_n}V_\alpha$,则称$V$为$Z_n$阶化向量空间。$V_\alpha$子空间中的元素称为$\alpha$阶齐次元。
\end{definition}
这个定义只要求这样的分解存在就行,并不需要搞清楚$V_\alpha$到底谁是谁。
\begin{theorem}
	若$V,W$都是$Z_n$阶化向量空间则它们上面的线性算子$\mathscr{L}(V,W)$也是一个$Z_n$阶化向量空间。张量积空间亦然。
\end{theorem}
考虑$V$上的双线性函数$f:V\times V\to K$,可以有下面几种形式:
\begin{description}
	\item[相容的:]$\forall x\in V_\alpha,y\in V_\beta,f(X,y)\propto\delta_{\alpha\beta}$
	\item[超对称的:]$\forall x\in V_\alpha,y\in V_\beta,f(X,y)=(-1)^{\alpha\cdot\beta}f(y,x)$
	\item[反超对称的:]$\forall x\in V_\alpha,y\in V_\beta,f(X,y)=-(-1)^{\alpha\cdot\beta}f(y,x)$
\end{description}
\begin{definition}
	超代数是一个$\mathbb{Z}_2$阶化的代数
\end{definition}
原先的李代数的代数乘法只有对易子,现在李超代数的目的就是把反对易子也加进来,而前面先定义阶化的目的就是分清楚哪些元素之间反对易关系,哪些之间是对易关系。
\begin{definition}
	李超代数是一个超代数,且乘法$(\cdot,\cdot)$满足:
	\begin{itemize}
		\item[双线性:] $(\alpha,k_1\beta_1+k_2\beta_2)=k_1(\alpha,\beta_1)+k_2(\alpha,\beta_2)\quad(k_1\alpha_1+k_2\alpha_2,\beta)=k_1(\alpha_1,\beta)+k_2(\alpha_2,\beta)$
		\item[阶化反对易:]$\forall x\in V_\alpha,y\in V_\beta,(x,y)=-(-1)^{\alpha\cdot\beta}(y,x)$
		\item[阶化Jacobi恒等式:]$\forall x\in V_\alpha,y\in V_\beta,\left((x,y),z\right)=\left(x,(y,z)\right)-(-1)^{\alpha\cdot\beta}\left(y,(x,z)\right)$
	\end{itemize}
\end{definition}

超代数是代数直接阶化,但是李超代数并非直接把李代数阶化。和李代数一样,李超代数的李超括号可以直接由超代数的乘法诱导:
$$[x,y]=xy-yx,\quad\{x,y\}=xy+yx$$
超李代数的生成元分为两组$x_i\in\mathfrak{g}_{\bar 0},y_\alpha\in\mathfrak{g}_{\bar 1}$,则对易关系可以写为:
\begin{equation}
	[x_i,x_j]=f^k_{ij}x_k,\quad [x_i,y_\alpha]=F^\beta_{i\alpha}y_\beta,\quad \{y_\alpha,y_\beta\}=A^i_{\beta\alpha}x_i
\end{equation}
且结构常数满足:
\begin{equation}
	f^k_{ij}=-f^k_{ji},\quad F^\beta_{i\alpha}=-F^\beta_{\alpha i},\quad A^i_{\beta\alpha}=A^i_{\alpha\beta}
\end{equation}

实际物理上考虑的是它们的射影表示,所以就跟CFT里面的Virasoro代数一样可以有个中心扩张。虽然Poincar\'e代数的中心扩张总可以消除,但是超Poincar\'e代数在$\mathcal{N}>1$时可以有非平庸的中心荷。李超代数的中心扩张定义和李代数完全一致,只是$c$和其他生成元不是对易,而是超对易,也就是$(c,x)=0,\forall x\in\mathfrak{g}$。

\begin{example}[一般线性李超代数]
	李代数比较重要的是考虑其到$\mathfrak{gl}(n)$的子代数的同态,也就是考虑其线性表示。李超代数这边同样也会考虑表示,这个时候对应的是$\mathfrak{pl}(m,n)$的子代数。考虑任意一个$\mathfrak{gl}(m+n)$中的元素,在某个基底下:
	\begin{equation}
		\begin{pmatrix}A_{m\times m}&B_{m\times n}\\C_{n\times m}&D_{n\times n}\end{pmatrix}=\begin{pmatrix}A_{m\times m}&0\\0&D_{n\times n}\end{pmatrix}+\begin{pmatrix}0&B_{m\times n}\\C_{n\times m}&0\end{pmatrix}
	\end{equation}
	这实际上就已经把$\mathfrak{gl}(m+n)$进行了$\mathbb{Z}_2$阶化,分块对角的是$\mathfrak{pl}(m,n)_{\bar 0}$,斜分块对角的是$\mathfrak{pl}(m,n)_{\bar 1}$。然后在矩阵乘法诱导的$[,]$和$\{\}$下构成李超代数。注意到对于同样一个$\mathfrak{gl}(m+n)$,可以有不同的$\mathbb{Z}_2$阶化方法,所以对于李超代数线性表示空间的维数需要用两个指标$m,n$来标记。
	
	常常对于$\mathfrak{pl}(m,n)$中的元素定义超迹:
	\begin{equation}
		\mathrm{str}\  x\equiv\tr A-\tr D
	\end{equation}
	它与基底选取无关,而且超迹自然诱导出一个$\mathfrak{pl}(m,n)\times\mathfrak{pl}(m,n)\to K$的双线性$\left<x,y\right>\equiv\mathrm{str}\ (xy)$,而且它是相容的,还是不变的,也就是说$\left<(x,y),z\right>=\left<x,(y,z)\right>$。超对易子的超迹必然是0,$\mathrm{str}\ (x,y)=0$。利用超迹还可定义出$\mathfrak{pl}(m,n)$的子代数$\mathfrak{spl}(m,n)\equiv\{x\in\mathfrak{pl}(m,n)|\mathrm{str}\ x=0\}$
\end{example}


李超代数和李代数有许多相似之处,比如同样可以定义单李超代数,可以考虑上面的Killing型和伴随表示,还可以考虑典型代数,用Dykin图和Cartan子代数方法找最高权表示或是对单李超代数进行分类。但是它又有不少不同点需要注意,比如Schur引理不再成立:
\begin{theorem}[超Schur引理]
	设$\rho:\mathfrak{g}\longrightarrow\mathfrak{pl}(V)$是李超代数的一个不可约表示,则与全部 $\rho(\mathfrak{g})$ 可超交换的线性变换,
	$$C(\rho)=\{a\in pl(V)|(a,p(\mathfrak{g}))=0\}$$
	只可能是 $C(\rho)=cI$, 或
	\[C(\rho)=c\begin{pmatrix}I_{n\times n}&b\\b^{-1}&I_{n\times n}\end{pmatrix},\quad c\in K\]
	这种情况要求$n=\dim V_{\bar 0}=\dim V_{\bar 1}$
\end{theorem}
再比如李代数的李定理也不再成立。更多有关李超代数表示论的内容请见\cite{superlie,WLXJ198301002,Kac:1977em}。