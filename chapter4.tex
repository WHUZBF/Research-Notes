\chapter{AdS/CFT对偶}
\section{大$N$极限}
本节的主要思想在\ref{B.5}中已经介绍不少了,本节目的是把目光局限在$SU(N)$ Y-M理论,这一话题的绝佳参考资料是\href{https://www.damtp.cam.ac.uk/user/tong/gaugetheory.html}{David Tong的规范场论讲义}对应章节。
\subsection{复习规范场}
首先随便\footnote{实际上这一选取不能是随便的,需要有些限制,详见Gelis\cite{Gelis:2019yfm}。}选一个规范群$G$。然后再加入一些标量场或者旋量场,它们处于$G$的某个表示之中,也就是说,在:
\begin{equation}\label{4.1.1}
	\phi_i(x)\mapsto U_{ij}\phi_j(x),\quad U_{ij}\in G
\end{equation}
的场位形变换下$\phi_i$(略去了旋量指标)的作用量应该保持不变,理论具有$G$的内禀对称性。现在我们把$U_{ij}$换成一个local的东东,也就是把每个元素都换成一个和空间位置有关的函数。显然我们利用$\phi_i$构造的拉氏量$\mathcal{L}(\partial\phi,\phi)$不一定在local的规范群下是不变的,罪魁祸首就是$\partial$,现在还要多出$\partial U$的项。可以考虑引入一个规范场$A^\mu$,在规范群的作用下如此变换:\footnote{其实就是在说其处于自伴表示下}
\begin{equation}\label{4.1.2}
	A_\mu(x)\mapsto U(x)A_\mu (x)U^\dagger (x)+\frac i g U(x)\partial_\mu U^\dagger(x)
\end{equation}
注意这里$A_\mu$是一个$N\times N$的矩阵形式的场,我们略去了矩阵指标,它也可以在生成元$T^a$基底下展开为$A^a T^a$。如果我们做下面的替换:
\begin{equation}
	\mathcal{L}(\partial\phi,\phi)\mapsto\mathcal{L}(D\phi,\phi),\quad D_\mu\equiv \partial_\mu -igA_\mu
\end{equation}
那么可以证明$\mathcal{L}$就在local规范群的作用下也是不变的,也就是说对场位形进行变换\ref{4.1.1}和\ref{4.1.2}导致的物理是不变的,而场位形本身不是可测的,我们关心的是场的激发导致的散射振幅。这意味着场位形空间是存在冗余的,这些由规范群联系起来的场位形应该看作是同一个场,路径积分也只用积一次,这直接导致了鬼场作为一个辅助场的存在。\footnote{这里有点微妙,$\phi_i$路径积分里面还是需要积分所有场位形,因为我们应该把\ref{4.1.1}看作是规范场变换\ref{4.1.2}诱导出来的一个被动的变换,所以一旦我们通过FP量子化方法取定了$A$的一个规范,那么用$U$联系的$\phi$也要看成不同。就像是现在把$A$当成横轴,$\phi$当纵轴,把$y=kx,k\in\mathbb{R}$线上的$(A,\phi)$看成等同,为了消除冗余,可以先固定$A$,那么$\phi$依然是自由的,每个$\phi$给出一个物理上独立的场位形$(A,\phi)$。}
\begin{remark}
	说到这里还是要多说一点,global的对称性和local的规范对称性看起来都是对场操作一下后$\mathcal{L}$不变,或者差个全导数。但是前者是真正的动力学对称性,也就是说可以用Noether定理推出非平凡守恒荷的,但是后者仅仅只是数学上描述的冗余,why?因为我们讲动力学对称性,话题的主角不是场位形本身,而是场的激发产生的物理态,动力学对称性就是说对态进行变换后得到一个新的态,两者的运动方程形式完全一致,是一样的动力学演化。比如中心势场问题,把物体的运动轨迹旋转一下,物理上可区分这两个不同的态,但是它们遵循相同的演化方程。而local的规范对称性只是数学上的冗余,只是拉氏量本身的对称性,没有任何的动力学意义,也就是说变换前后两个态就是物理上无法区分的,而且不会带来非平凡守恒荷。所以前者我们场位形需要全积,后者我们只需要积某个特定规范下的场位形。当然,在无穷远处不归0的规范变换实际上是真正的对称性,即所谓渐近对称性,这个就不多谈了,具体可见A.Strominger的工作。
\end{remark}

规范场有两种视角来看,第一种就是因为我们要求的散射振幅是$\left<A^aA^b\cdots\right>$这些带色指标的分量场构成的关联函数(前面还要乘上对应的极化矢量)。所以可以直接用$A^a$来写费曼规则。第二种就是由于$A^a=\tr \left(A T^a\right)$,所以我们可以先把$A^\mu$就看作一个独立的矩阵,直接从矩阵来写费曼规则,最后原则上散射振幅就能用$\left<{A^{i}}_j{A^{k}}_l\ldots\right>$求迹得到。类似于\ref{B.5}中的double line表示,对于和一个费米子耦合的YM理论:
\begin{equation}
	1
\end{equation}
有:
\begin{equation}
	1
\end{equation}
所以费曼图为:
\begin{equation}
	1
\end{equation}
我们忽略了鬼场的影响,不过就后面讨论的目的而言问题不大。